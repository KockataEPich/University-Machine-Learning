\documentclass{article}

\title{COMP26120 Lab 5}
\author{?}

\begin{document}
\maketitle

% PART 2 %%%%%%%%%%%%%%%%%%%%%%%%%%%%%%%%%%%%%%%%%%%%%%%%%%%%%%%%%%%%%%%%%%%%%%

\section{Complexity Analysis}
\label{sec:complexity}

\subsection{Iteration Sort}
Best case senario for insertion sort is to literally be given a sorted array, in which case
the sort iterates all of the items, giving a time complexity of n; \\

  Worst case senario happens when the array is sorted in reverse order.
Then, we have to iterate all of the elements (n), however with each of the bigger iterations, we have to have $(1 + 2 + 3 + ... + (n - 1))$ smaller iteration for the actual sorting. So when we combine the two we get: \\

   $n(1 + 2 + 3 + ... + (n-1)) 
=  1n + 2n + 3n + ... + n(n-1))
=  1n + 2n + 3n + ... + n^2 - n$

 which has n2 complexity \\

where $x[i]$ represents the number of second ¨for¨ loops which need to be done for a particular
element and $x[i] \in [0:n-2] $


 The avarage case senario has the equation \\

$   n(x1 + x2 + x3 + ... + (n-Xn-1))
=  x1n + x2n + x3n + ... + n(n-Xn-1))
=  x1n + x2n + x3n + ... + n2 - (Xn-1)n $ \\

 which still has $n^2$ complexity 
This makes the insertion sort inefficient when dealing with large files. \\



\subsection{Quick Sort}

  Best case senario occurs when the partitions are evenly balanced and the pivot is right in the middle
after partitioning. \\

In this case we have
$cn + cn/2 + cn/4 + cn/8 + ...$
which has $n\log_{2}n$ \\

  Worst case senario occurs when the partitions are the most unbalanced (we have to partition every single element separately).
$cn + c(n-1) + c(n-2) + ... + 2c$ \\

  The avarage case senario occurs when each pivot is between the 1/4 to 3/4 of the overall lenght with
3/4 of the elements on the right and 1/5 on the left of the pivot. Applyng this rule we get 
$n\log_{2}n$

%====================================
\section{Experimental Analysis}
\label{sec:initialExperiments}

In this section we consider the question
	\begin{quote}
	Under what conditions is it better to perform linear search rather than binary search?
	\end{quote}

\subsection{Experimental Design}

\subsection{Experimental Results}


% PART 3 %%%%%%%%%%%%%%%%%%%%%%%%%%%%%%%%%%%%%%%%%%%%%%%%%%%%%%%%%%%%%%%%%%%%%%

\section{Extending Experiment to Data Structures}
\label{sec:part3}

We now extend our previously analysis to consider the question
\begin{quote}
Under what conditions are different implementations of the dictionary data structure preferable?
\end{quote}

% PART 3 %%%%%%%%%%%%%%%%%%%%%%%%%%%%%%%%%%%%%%%%%%%%%%%%%%%%%%%%%%%%%%%%%%%%%%
\section{Conclusions}
\label{sec:conclusions}
% Give your conclusions from the above experiments 


\end{document}
